%% Language-Based Security project proposal template
%% 2022-02-07
%% To use this template:
%% - Remove \instruction commands and fill-in with your text
%% - Include your names in the author command
%%
%% - Compile with Makefile
%% - Edit literature.bib to include relevant literature.
%%
%% - Submit _only_ the resulting PDF file.
%%


\documentclass[10pt]{article}
\usepackage[left=0.8in,right=0.8in,bottom=0.5in,top=0.5in]{geometry}
\geometry{a4paper}
\usepackage[parfill]{parskip}    % Activate to begin paragraphs with an empty line rather than an indent
\usepackage{graphicx}
\usepackage{xcolor}
\usepackage{hyperref}
%\usepackage{times}
\usepackage{titling}
\usepackage[small,compact]{titlesec}


% \renewcommand{\familydefault}{\sfdefault}


\usepackage{amsmath,amssymb}
\DeclareGraphicsRule{.tif}{png}{.png}{`convert #1 `dirname #1`/`basename #1 .tif`.png}



\newcommand\courseName{Language-Based Security}
\newcommand\courseNameAbbrv{LBS}
\newcommand\courseYear{2023}
\newcommand\courseAndYear{\courseNameAbbrv-\courseYear}
\newcommand\reportKind{Project report}

\newcommand\groupNumber[1]{
  \makeatletter
  \def\@courseGroupNumber{#1}
  \makeatother
}

\pretitle{\begin{flushright} \bfseries  \large \courseAndYear: \reportKind \end{flushright}   \begin{flushleft}
\bfseries \LARGE}
\posttitle{\par\end{flushleft}\vskip 0.5em}

\preauthor{
\begin{flushleft}
\large \lineskip 0.5em%
\bfseries
\makeatletter Group \@courseGroupNumber \\ \makeatother
\begin{tabular}[t]{@{}l}}
\postauthor{\end{tabular}\par\end{flushleft}}

\predate{\begin{flushleft}\bfseries \large}
\postdate{\par\end{flushleft}}

\newcommand\instruction[1]{{\ttfamily{\color{red}{#1} }}}


\usepackage{mathpartir}
\usepackage{pftools}
\usepackage{stmaryrd}
\usepackage{amsmath}
\usepackage{amsthm}
\usepackage{fancyvrb}

\usepackage{tikz-cd}
\usetikzlibrary{cd}

% Notations
\newcommand{\pc}{\mathrm{pc}}
\newcommand{\lpc}{\mathrm{lpc}}
\newcommand{\ctx}{\Gamma}
\newcommand{\conf}{\sigma}
\newcommand{\public}{\mathrm{LPublic}}
\newcommand{\secret}{\mathrm{LSecret}}
\newcommand{\typing}[4]{ #1,~#2 \vdash #3 \rightsquigarrow #4}
\newcommand{\typingaux}[5]{ #1,~#2 \vdash #3 \rightsquigarrow #4,~#5}
\newcommand{\execaux}[7] { (#1,~#2,~#3) \xrightarrow{#4} (#5,~#6,~#7) }
% (config, Γ, lpc) - ev -> (config_f, Γf, lpcf)
\newcommand{\exec}[2] { #1 \rightarrow #2 }
\newcommand{\exectrans}[2] { #1 \rightarrow^{*} #2 }
\newcommand{\execn}[3] { #1 \rightarrow^{#2} #3 }
\newcommand{\bridge}[3] { #1 \curvearrowright_{#2} #3 }
\newcommand{\whilelang}{\textsc{While Lang} }
\newcommand{\pproj}[1]{\llbracket #1 \rrbracket_{p}}
\newcommand{\agree}[3]{#2 \sim_{#1} #3}
\newcommand{\compile}[1]{#1\!\downarrow\ }
\newcommand{\uncompile}[1]{#1\!\uparrow\ }

\newcommand{\join}[2]{#1 \sqcup #2}
\newcommand{\flowsto}[2]{#1 \sqsubseteq #2}
\newcommand{\etyping}[3]{ #1 \vdash #2 : #3}
\newcommand{\pini}[1]{\mathrm{PINI}(#1)}

% Environment
\newtheorem{theorem}{Theorem}
\newtheorem{lemma}{Lemma}

\title{
   Flow-Sensitive Type System for PINI
}

\groupNumber{1}
\author{Maxime Legoupil \and Bastien Rousseau }

\begin{document}
\maketitle

% \begin{abstract}
%    Abstract of the project report: what, why PINI ? Type system for enforcement. Soundness.
% \end{abstract}

\thispagestyle{empty}

\section{Introduction}
\label{sec:intro}

In a computer system, the security policies define what kind of properties are desired on the
system. Integrity, confidentiality and availability are widespread and well-studied policies.
The security mechanisms are the implementations of a specific method to enforce the security policy,
against an attacker that have a precise, specific power over the system.

In the case of confidentiality, we want to keep some data of the system secret. An attacker should
not be able to read, or get any information about these data. We consider a simple system, a
programming language with I/O primitives, communicating with two kinds of I/O channels: public and
secret channels.
The security policy that we want to ensure is the confidentiality of the secret inputs. We
want to make sure that the secret data are not leaked. To enforce the confidentiality, we define a
type-system enforcing the values typed as secret are not printed through the public channel.
The threat model is that an attacker can observe the public channels of any well-typed program.
Side-channel such as timing-channel or cache-channel are out of scope.

In our security model, we assure that the type-system actually enforce confidentiality using a
formal proof.
The first step is to formally define what \emph{confidentiality} means. A widely used
definition of confidentiality is called \emph{non-interference}. In a high-level way,
non-interference means that ``public outputs do not depend on secret inputs''.
More formally, non-interference is a 2-hyperproperty, which means that it is a property that
depends on 2 execution traces of the program. It captures the confidentiality by comparing
two execution traces with the same public inputs, but different secret inputs. If it is possible
to differentiate the traces by looking only at the public output events, only the secret inputs can
be responsible (as the public ones are the same in both execution).

Non-interference (NI) is actually a class of properties for which it exists different variants. For
instance, we can cite \emph{Termination Sensitive NI} (TSNI) or \emph{Termination Insensitive NI}
(TINI). The latter allows an adversary to differentiate a secret with the non-termination of the
execution, while the former does not.
There are traditionally two ways to enforce non-interference:
\begin{itemize}
  \item monitor based: \emph{dynamically} checks that the public output channel does not leak values
        tagged as secret
  \item type-system based: \emph{statically} ensures that a well-typed program respects NI
\end{itemize}

In this work, we focus in a variant of NI called \emph{Progress-Insensitive Non-Interference}
(PINI).
Intuitively, PINI ensures that, if the adversary knows that there will be at least one more public
event, it cannot gain any knowledge on the secret by learning what this event actually is. PINI is
interesting, as we can see it as a variant of TINI with intermediate I/O.
We develop a flow-sensitive type-system that enforces PINI, ie. all well-typed program respects PINI,
and we prove the type system to be sound.

In the following report, we first introduce in
\S\ref{subsec:opsem} the I/O \whilelang language, and its operational semantic.
We then formally define in \S\ref{subsec:ni} the notion of Progress-Insensitive
Non-Interference, and we presents the type-system that enforces it in
\S\ref{subsec:type_system}.
Finally, we detail the soundness theorem and its proof outline in \S\ref{sec:soundness}.

This work is partially formalized in the Coq Proof Assistant, available in a
\href{https://github.com/BastienRousseau/flow-sensitive-PINI}{Github repository}.
The proof is not complete, and we develop the limits of our implementation and the difficulties
that we encountered in \S\ref{sec:conclusion}.

\section{Formalization}
\label{sec:formalization}
In this section, we formally introduce the \whilelang, the non-interference property and the
type-system.

\subsection{\whilelang}%
\label{subsec:opsem}

\whilelang is a simple WHILE language with input and output primitives. Figure~\ref{fig:lang_syntax}
shows the syntax of the language.

\begin{figure}[ht]

$e ::= n\ |\ x\ |\ e \oplus e$

$c ::=
\mathrm{Skip}\
|\ \mathrm{Assign}~x~e\
|\ \mathrm{If}~e~c~c\
|\ c;c\
|\ \mathrm{While}~e~c\
|\ \mathrm{Input}~ch~x\
|\ \mathrm{Output}~ch~e\
$

$ch ::= \mathrm{Public}\ |\ \mathrm{Secret}$

\caption{\label{fig:lang_syntax} Syntax of \whilelang}
\end{figure}


A configuration is a tuple $\conf \triangleq (c,~S,~P,~m,~\tau)$, where $c$ is a command, $S$ and
$P$ are resp. the secret and public input streams, $m$ is the memory (a partial map
from variables to values) and $\tau$ is the trace of events emitted so far.
For the sake of the presentation, we might use $c$ and $\conf$ interchangeably when it is clear from
the context.

Figure~\ref{fig:lang_opsem} in the Appendix shows the operational semantic of the language. The operational semantics
are as expected, so we elide extended explanations.

\subsection{Non-interference}%
\label{subsec:ni}

The mantra of non-interference is that ``public outputs do not depend on secret inputs''.
We will define formally this concept using the notion of \emph{attacker knowledge}. The attacker
knowledge is the ability for an adversary to deduce the possible secret inputs. It characterizes
what an attacker can learn by looking at the public outputs.
More precisely, the attacker knowledge is the set of secret inputs that are consistent with the
public execution trace. Thus, smaller is the set of possible inputs, bigger is the attacker
knowledge.

Formally, the attacker knowledge $K$ is defined by
\[
  K(P,~c,~\delta) \triangleq
  \{ S ~|~ \exists \conf.\
  \exectrans{(c,~S,~P,~m_{\mathrm{init}},~\varepsilon)}{\conf}
  \wedge
  \pproj{\tau} = \delta
  \}
\]
where $\conf = (c',~S',~P',~m',~\tau)$ and $\pproj{\tau}$ is the projection of the public events of
the trace $\tau$.

Using this base definition, we can define a variant of NI called \emph{Progress Sensitive
  Non-Interference} (PSNI).
\begin{align*}
  \mathrm{PSNI} &\triangleq \\
  &(\exists c',~S',~P',~m',~\tau.\
  \exectrans{(c,~S,~P,~m_{init},~\varepsilon)}{(c',~S',~P',~m',~\tau)}
  \wedge
  \pproj{\tau} = \delta) \\
  &\Rightarrow
  \ K(P,~c,~\alpha::\delta) = K(P,~c,~\delta)
\end{align*}
The intuition is that, if the adversary has already discovered the public trace $\delta$ during the
execution, discovering a new public event $\alpha$ does not provide more knowledge.
In other words, at any step of computation, the knowledge of the attacker is the same, and it cannot
narrow down the set of possible secret inputs.
However, PSNI is hard to enforce because we need to know whether there will be a future event, and
loops are a challenge in this task.

Because PSNI is hard to enforce, we define the \emph{Progress Insensitive Non-Interference} (PINI)
variant. Intuitively, it allows the adversary to know that there will be a future public event, but
when it eventually learns the exact event, it does not change its knowledge. In order to formally
define this intuition, we first define the notion of \emph{progress knowledge} $K_{\rightarrow}$,
which is a variant of the attacker knowledge with an additional (any) public event.
\[
  K_{\rightarrow}(P,~c,~\delta) \triangleq
  \{ S ~|~ \exists \conf',\alpha.\
  \exec{(c,~S,~P,~m_{\mathrm{init}},~\varepsilon)}{\conf'}
  \wedge
  \pproj{\tau'} = \alpha::\delta
  \}
\]
The progress knowledge captures the set of secret inputs that are consistent with the execution $c$,
the already known public trace $\delta$ and any additional public event $\alpha$.
We can thus formally define PINI:
\begin{align*}
  \mathrm{PINI} &\triangleq \\
  &(\exists c',~S',~P',~m',~\tau.\
  \exectrans{(c,~S,~P,~m_{init},~\varepsilon)}{(c',~S',~P',~m',~\tau)}
  \wedge
  \pproj{\tau} = \alpha::\delta) \\
  &\Rightarrow
  \ K(P,~c,~\alpha::\delta) = K_{\rightarrow}(P,~c,~\delta)
\end{align*}
which indeed captures the fact that learning an additional event does not give more knowledge than
knowing that a public event will arrive.
PINI is easier to enforce than PSNI, because the attacker already knows in advance that there will
be another event, but it does not know which one precisely. In some way, it does not get surprised
by the upcoming event.

\subsection{Type system}%
\label{subsec:type_system}
In this section, we define a flow sensitive type-system that enforces PINI.

The typing judgment for expression is fairly standard. \( \etyping{\ctx}{e}{\ell} \) means that
the expression \( e \) has confidentiality level \( \ell \) in context \( \ctx \).
Figure~\ref{fig:typing_expr} shows the typing rules for expressions. Any binary operations is typed
with the join of the labels, even though this could be made more precise (eg. the expression $x - x$ is
always 0, so it could be public regardless of the security level of $x$).

\begin{figure}[ht]
  \centering
   \begin{mathparpagebreakable}
    \inferH{Lit}{ }
    {\etyping{\ctx}{\mathrm{Lit}~n}{\public}}
    \and

    \inferH{Var}
    {\ctx(x) = \ell}
    {\etyping{\ctx}{\mathrm{Var}~x}{\ell}}
    \and

    \inferH{BinOp}
    {\etyping{\ctx}{e_{1}}{\ell_{1}} \and
      \etyping{\ctx}{e_{2}}{\ell_{2}}}
    {\etyping{\ctx}{e_{1} \oplus e_{2}}{\join{\ell_{1}}{\ell_{2}}}}

   \end{mathparpagebreakable}
  \caption{\label{fig:typing_expr} Typing rules for expressions }
\end{figure}


The typing judgment for the commands \( \typing{\ctx}{\pc}{c}{\ctx'} \), means that under the context
\( \ctx \) and execution level \( \pc \), full execution of command \( c \) will modify the context to finally be $\ctx'$.

Figure~\ref{fig:typing_cmd} presents the typing rules for the commands.
\begin{figure}[ht]
  \centering
 \begin{mathparpagebreakable}
    \inferH{Skip} {\flowsto{\ctx}{\ctx'} }
    {\typing{\ctx}{\pc}{\mathrm{Skip}}{\ctx'}}
    \and

    \inferH{Assign}
    { \etyping{\ctx}{e}{ \ell } \and \flowsto{(x:\join{\ell}{\pc})::\ctx}{\ctx'}
    }
    {\typing{\ctx}{\pc}{\mathrm{Assign}~x~e}{\ctx'}}
    \and

    \inferH{Seq}
    {
      \typing{\ctx_{1}}{\pc}{c_{1}}{\ctx_{2}} \and
      \typing{\ctx_{2}}{\pc}{c_{2}}{\ctx_{3}}
    }
    {\typing{\ctx_{1}}{\pc}{c_{1};c_{2}}{\ctx_{3}}}
    \and

    \inferH{If}
    { \etyping{\ctx}{e}{ \ell } \and
      \typing{\ctx}{\join{\pc}{\ell}}{c_{1}}{\ctx_{1}} \and
      \typing{\ctx}{\join{\pc}{\ell}}{c_{2}}{\ctx_{2}} \and
    }
    {\typing{\ctx}{\pc}{\mathrm{If}~e~c_{1}~c_{2}}{\join{\ctx_{1}}{\ctx_{2}}}}
    \and

    \inferH{While}
    { \flowsto{\ctx}{\ctx'} \and
      \etyping{\ctx'}{e}{ \ell } \and
      \typing{\ctx'}{\join{\pc}{\ell}}{c}{\ctx'}
    }
    {\typing{\ctx}{\pc}{\mathrm{While}~e~c}{\ctx'}}
    \and

    \inferH{Input}
    { \flowsto{\pc}{ch} \and \flowsto{(x:\join{\ell}{\pc})::\ctx}{\ctx'}}
    {\typing{\ctx}{\pc}{\mathrm{Input}~ch~x}{\ctx'}}
    \and

    \inferH{Output}
    { \etyping{\ctx}{e}{\ell} \and
      \flowsto{\join
        {\pc}{\ell}}{ch} \and \flowsto{\ctx}{\ctx'}}
    {\typing{\ctx}{\pc}{\mathrm{Output}~ch~e}{\ctx'}}
 \end{mathparpagebreakable}
  \caption{\label{fig:typing_cmd} Typing rules for commands }
\end{figure}


    % \inferH{While1}
    % { \etyping{\ctx}{e}{ \ell } \and
    %   \typing{\ctx}{\join{\pc}{\ell}}{c}{\ctx}
    % }
    % {\typing{\ctx}{\pc}{\mathrm{While}~e~c}{\ctx}}
    % \and

    % \inferH{While2}
    % { \etyping{\ctx}{e}{ \ell } \and
    %   \typing{\ctx}{\join{\pc}{\ell}}{c}{\ctx''} \and
    %   \typing{\ctx''}{\join{\pc}{\ell}}{c}{\ctx'}
    % }
    % {\typing{\ctx}{\pc}{\mathrm{While}~e~c}{\ctx'}}
    % \and


The type system captures the modifications of the typing context after the execution of the command.
In particular, it captures the programs that change the security level of a variable. Because the
observable events are the outputs of the public channel, the internal modifications of the memory are
not observable right away. As such, the confidentiality level of the variables in memory can safely
be modified.

An interesting rule is the \textsc{While}. It allows to choose the final context $\ctx'$, that will
be used to type both the expression and the body of the loop. This final context must flows from the
initial context $\ctx$.
$\ctx'$ captures the strictest context $\ctx'$, to type the expression and the
body of the loop, as it has to be the same before and after body. In some way, it expresses some
sort of a fixpoint over the possible typing-context for the body. It is also necessary to type the
expression with the final context, as the type of the expression can depend on variables that are
modified in the body.

The rest of the typing rules are quite standard information-flow type system. Now that we have
defined a type-system, we want to prove that it is non-trivial (it types interesting programs) and
sound (it captures the wanted property). We refer the reader to the
Appendix~\ref{sec:examples} for some examples of well-typed programs, and we show the soundness in
the next section.

\section{Soundness}%
\label{sec:soundness}

The soundness theorem says that the type system enforces PINI. In other words, it formally states
that any well-typed program is PINI. A state is well-formed when the command types in a context \( \Gamma \) and the initial memory is well-formed wrt. \( \ctx \), which we define by 
%We say that a memory $m$ is well-formed wrt. a context $\ctx$, if
all the variables defined in $m$
are also defined in $\ctx$, ie. $\forall x, x \in m \implies x \in \ctx$.

\subsection{Approach}%
\label{subsec:approach}

To prove that all well-typed programs are PINI, we suppose we have two runs of the program with observable traces of equal length. The high-level intuition of the proof is that ``at every step that produces a public event, the states are observationally equivalent''.

The way we define ``at every step that produces a public event'' is similar to the approach of (information flow lecture notes), via use of a bridge relation (or in fact several bridge relations), as detailed in \S\ref{subsec:bridge_rel}.

The we we define states to be ``observationally equivalent'' is more interesting. This entails, for one, that we should define what it means for two memories to be observationally equivalent. To that end, 
we say that a two memories $m_{1}$ and $m_{2}$ agree on the public variables wrt. a context $\ctx$,
if all the variables defined as public in $\ctx$ are the same in $m_{1}$ and $m_{2}$. We note
$\agree{\ctx}{m_{1}}{m_{2}}$. Knowledge of the context \( \ctx \) is crucial, as our operational semantics allows the context to change over execution if, say, a variable that carried a secret value is assigned a new public value.

But this causes a difficulty: what context to use when proving that, mid-execution, two memories agree on public variables? To this end, we introduce a new augmented operational semantics, that keeps track of the context. In order to be able to correctly update the context when executing an assignment, we also need to keep track of the execution level \( \pc \), and this execution level can change as well when executing if-then-else expression. Hence at first glance, we could try to define our new augmented semantics in this form:
\[ (\sigma, \ctx, \pc) \to (\sigma', \ctx', \pc') \]

However, defining this augmented semantics on the language as it was introduced earlier is tricky, since when executing an if-then-else expression and modifying the execution level \( \pc \), we must be able to know when we are done executing the branch inside the if-then-else and should return to the previous \( \pc \).

The solution to this difficulty is to imitate what is shown briefly in (information flow lecture notes): add a \( \mathrm{Join} \) instruction, and keep track not of the current level \( \pc \), but of a list \( \lpc \) of all the levels of the conditions of the ifs we are under. When executing an if-then-else, we add the level of the condition to the list \( \lpc \), and when encountering a \( \mathrm{Join} \), we remove the last security level from the list \( \lpc \). Thus our augmented semantics actually looks like this:\footnote{plus a \emph{public event} on top of the arrow, as we will show in \S\ref{subsec:bridge_rel}}.
\[ (j\sigma, \ctx, \lpc) \to (j\sigma', \ctx', \lpc') \]
with \( j\sigma \) representing a \emph{jconfiguration}, i.e. a configuration with a \emph{jcommand} instead of a command. We define jcommands and the augmented semantics formally in \S\ref{subsec:aux_lang}
%But first, let us define two notions that relate memories and context.





%\subsection{Approach}%
%\label{subsec:approach}

%In order to prove the theorem, we follow a similar approach than the one in (information flow
%lecture notes).

%First, we define an auxiliary, augmented language, that explicitly specify when we reach a join
%point. Then, we define its semantic, that keeps track of the type with the typing environment and
%the current security level.

Then, we show that jcommands are somewhat equivalent to the original language, in the
sense that there is a way to compile commands into jcommands such that reduction steps in the augmented semantics reflect reduction steps in the original semantics. As such, we can reason on jcommands,
and reflect the result on the original one with an adequacy theorem.

Finally, the main proof uses a bridge relation that allows to reason precisely on the
transitions emitting a public event, and follows the high-level idea introduced above.

\subsection{Auxiliary language}%
\label{subsec:aux_lang}

% Syntax and opsem
Let us define the auxiliary language of \emph{jcommands} with the exact same syntax as the WHILE language introduced earlier, but we the added command
\emph{join}. Having an explicit Join\footnote{  In the Coq development, we add a construct \( \mathrm{ThenJoin}\ j \) instead of a \( \mathrm{Join} \). This is meant to represent the program \( j;\ \mathrm{Join} \), since as we will see, all compiled commands \( \compile{c} \) always have the Joins as the right-hand-side of a sequence. This phrasing makes it easier to define \( \uncompile{j} \) and makes the equivalence lemmas between jcommands and commands easier to state and to prove. In fine, both approaches are equivalent. In the appendix, we show the rules used for the \( \mathrm{ThenJoin}\ j \) constructor, to reflect the contents of the Coq development.
} allows us to reason about
the security level of the pc when leaving a if-then-else branch.


\[ \mathit{j}~::=~\mathrm{Skip}\
|\ \mathrm{Assign}~x~e\
|\ \mathrm{If}~e~j~j\
|\ j;j\
|\ \mathrm{While}~e~j\
|\ \mathrm{Input}~ch~x\
|\ \mathrm{Output}~ch~e\ |~\mathrm{Join} \]

We define a translation from commands to jcommands by defining the jcommand \( \compile{c} \) as a morphism in all cases but If, where \[ \compile{(\mathrm{If}~e~c_1~c_2)}=\mathrm{If}~e~(\compile{c_1};\mathrm{Join})~(\compile{c_2};\mathrm{Join}) \] which reflects that we wish to join at the end of the branches of an If.

Conversely, we define \( \uncompile{j} \) as the command that is equal to \( j \) with all its Joins stripped.

As mentionned before, the augmented operational semantic carries along the typing context and a list of security labels corresponding to the security levels of all conditions of ifs that the current expression is under. We denote by \( j\conf \) the equivalent of configurations \( \sigma \) with a jcommand instead of a command.

The relation \(\execaux{j\conf}{\ctx}{\lpc}{\alpha}{j\conf_{f}}{\ctx_{f}}{\lpc_{f}} \)
states that the jconfiguration $j\conf$, with the typing context $\ctx$, at the current security level
$\lpc$ steps to the configuration $j\conf_{f}$, with the new typing context $\ctx_{f}$ and the new
security level $\lpc_{f}$, and emits the event $\alpha$, which can be empty. We will come back to this event in \S\ref{subsec:bridge_rel}.

The main feature of the operational semantic is the management of the pc: at every conditional
branch, the history is incremented with the security label of the condition expression to the pc.
When reaching a Join instruction, the last pc in the history is popped. If the history is empty,
the semantics is stuck, as Joins are meant to indicate that we previously entered an If.


%Moreover, the language of events is enriched with a Write event, that is emitted at every
%assignment of a public expression. We need to enrich the events because we need to keep track of the
%modifications in memory.
The full operational semantic rules are available in the appendix in
Figure~\ref{fig:lang_aux_opsem}.

We also define a typing judgment for the auxiliary language of jcommands, which refines the typing judgment of
commands.

The judgement \( \typingaux{\ctx}{\lpc}{j}{\ctx'}{\lpc'} \) states that jcommand \( j \) is well-typed in context \( \ctx \) and under guards \( \lpc \), and after its full execution, the context will be \( \ctx' \) and the list of guards will be \( \lpc' \). The apparition of \( \lpc' \) here is simply to reflect that in general, when executing an arbitrary jcommand \( j \), there might not be Join commands matching all Ifs. We can however show that for all command \( c \), if \( \typingaux{\ctx}{\lpc}{\compile{c}}{\ctx'}{\lpc'} \), then \( \lpc = \lpc' \).

First, the typing judgment captures the pc after the execution of the jcommand. Then, as the pc
is now an history of pc, all conditions over the pc takes the join of the history. The operation are
thus lifted to the history of pc.
Moreover, the typing rules for the auxiliary language are more relax than
the one for \whilelang, as it allows to chose a final typing context $\ctx'$ that flows from the
starting typing context $\ctx$.



The typing system is also extended with a rule for the join, which captures the fact that a Join
leaves the current security level by removing the head pc of the history.

\[
    \inferH{Join}
    {\ } %\typingaux{\ctx}{\lpc}{c}{\ctx'}{\pc::\lpc'}}
    {\typingaux{\ctx}{\pc::\lpc}{\mathrm{Join}}{\ctx'}{\lpc}}
  \]

The full typing rules are available Figure~\ref{fig:typing_aux}.

The auxiliary language and type-system enjoy some interesting properties that are detailed in the
remaining of this section.

% Preservation
The preservation lemma states that if a well-typed jcommand takes a step, then the resulting term is
well-typed, under the (potentially) new typing context.

\begin{lemma}[Preservation]\label{thm:preservation}
  Let $\conf \triangleq (j,~S,~P,~m,~\tau)$ and
  $\conf' \triangleq (j',~S',~P',~m',~\tau')$ be jconfigurations.
  Let us further suppose that:
  \begin{itemize}
    \item $m$ is well-formed wrt. $\ctx$
    \item $\typingaux{\ctx}{\lpc}{j}{\ctx_{f}}{\lpc_{f}}$
    \item \( \execaux{j\conf}{\ctx}{\lpc}{\alpha}{j\conf'}{\ctx'}{\lpc'} \)
  \end{itemize}

  Then, $m'$ is well-formed wrt. $\ctx'$, and the command $j'$ is also well-typed:
  $\typingaux{\ctx'}{\lpc'}{j'}{\ctx_{f}}{\lpc_{f}}$.
\end{lemma}

% Adequacy wrt. the WHILE language
%In order to establish an equivalence between the original language and the auxiliary language, we
%define a transformation from the auxiliary language to the original language, which consists on
%erasing the join command; and a transformation from the original language to the original language,
%which consists on adding the join command after every if-then-else.
%We denote the transformation from a command $c$ in \whilelang to the auxiliary language
%$\compile{c}$.

The equivalence between the two languages are reflected via the two following adequacy theorems:

\begin{lemma}[Adequacy auxiliary typing]
  For any $\ctx$, $\pc$, $c$ and $\ctx_{f}$,
  \[
  \typing{\ctx}{\pc}{c}{\ctx_{f}}
  \iff
    \typingaux{\ctx}{\lpc}{\compile{c}}{\ctx_{f}}{\lpc}
  \]

  for any \( \lpc \) such that \( \bigsqcup\lpc = \pc \).
\end{lemma}

The adequacy of the auxiliary typing establishes that the transformation from one language to
another preserves the typing judgment for the corresponding language.

\begin{lemma}[Adequacy auxiliary language]
  Let $\conf \triangleq (c,~S,~P,~m,~\tau)$ and
  $\conf' \triangleq (c',~S',~P',~m',~\tau')$ be configurations.
  Let us suppose that:
  \begin{itemize}
    \item $m$ is well-formed wrt. $\ctx$
    \item $\typing{\ctx}{\pc}{c}{\ctx_{f}}$
    \item $\exec{\conf}{\conf'}$
  \end{itemize}

  Then, there exists $j\conf'$, $\ctx'$, $\lpc'$ and $\alpha$ such that
  \( \execaux{\compile{\conf}}{\ctx}{[\pc]}{\alpha}{j\conf'}{\ctx'}{\lpc'} \) and \( \uncompile{j\conf'}=\conf' \)
\end{lemma}

The adequacy of the auxiliary language establishes that, if the memory is well-formed, the
command is well-typed in the original language, and takes a step, then we can also take a similar
step in the auxiliary language.

In the following diagram, the adequacy lemma states that $j'$ is the compilation of $c'$:

\begin{center}
\begin{tikzcd}
\compile{c} \arrow[r, dashed]  & j' \arrow[d, dashed]                  \\
c \arrow[u] \arrow[r] & c'
\end{tikzcd}
\end{center}

The auxiliary language allows us to reason on an equivalent language to \whilelang, but which
carries over more information. We are now ready to carry out the rest of the proof.

\subsection{Bridge relations}%
\label{subsec:bridge_rel}

When executing two runs from the same command with the same public inputs, the two runs may take different paths in some execution steps, for instance when executing an if-then-else statement whose condition is a secret expression. If the command is well-typed, typing guarantees that no public inputs or outputs occurs while the runs have different behaviours. In fact, the typing even guarantees that any assigns in these phases will be on secret variables, which means when exit the if-branch, the memory will agree on public variables, which is what gives the program the PINI property.

This motivates separating non-emitting execution steps from those who emit public events.
%Some step do not emit any event, which makes these steps
%irrelevant regarding the definition of the knowledge. As such, we want a relation that skips the
%irrelevant steps, and allows us to reason on the steps emitting a public event only.
We establish this by introducing \emph{bridging relations}.

In order to maintain the idea introduced in \S\ref{subsec:approach} that ``at every step that produces a public event, the states are observationally equivalent'', it is convenient to consider writes to public variables to be public events, as well as public variables becoming secret. To this end, the event \( \alpha \) mentionned in \S\ref{subse:aux_lang} in our augmented jcommand semantics are not only inputs and outputs, but also public writes \( \mathrm{Write}~x~v \) and variables ceasing to be public \( \mathrm{BecameSecret}~x \).


With this definition of public events, we can show that a single step in the augmented jcommand semantics
\begin{itemize}
\item does not observationally change the state if no public event is produced
\item does not observationally chaneg the trace if no public input or output is produced
\item gives observationally equivalent results when given observationally equivalent starting points, if a public event is produced
\end{itemize}

The definition of ``observationally equivalent'' is left vague here, as we have modified its exact meaning many times in order to adapt the proofs, and haven't settled on one that works. We believe that it would be sensible to require public input streams to be equal, require nothing of the secret input streams, require `equivalent' contexts, require memories to agree on public values in these equivalent contexts, and require the public projections of the traces to be equal. However, it appears that in some proofs, we would also need to relate the \( \lpc \) lists in the states, and it is unclear how to do that meaningfully.


We define multiple kind of bridging relations, that relates two states in different ways, depending
of the kind of public events.

\begin{itemize}
  \item Basic-bridge relation: % (\emph{basic\_bridge} in the Coq development):
        two states basic-bridge in n steps, if they take n-1 steps without emitting any public
        event, and then take a step with a public event.
  \item Write-bridge relation: % (\emph{write\_bridge} in the Coq development):
        two states write-bridge in n steps, if they take n-1 steps without emitting any public
        event, and then take a step with a \( \mathrm{Write}~x~v \) or \( \mathrm{BacemeSecret}~x \) public event.
  \item Silent-bridge relation: %(\emph{silent\_bridge} in the Coq development):
    two states silent-bridge in n steps, if they take n steps without emitting any public event
\end{itemize}

Finally, the \emph{bridge} relation ties everything together (here we conflate the notation \( j\sigma \) with the more correct \( (j\sigma, \ctx, \lpc) \)): \( j\sigma \) bridges to \( j\sigma' \), which we write \( \bridge{j\sigma}{\tau}{j\sigma'} \), where \( \tau \) is the sequence of public events produced, if either \begin{itemize}
\item there exists a state \( j\sigma_m \) such that \( j\sigma \) becomes \( j\sigma_m \) in 0 or more write-bridges, and \( j\sigma_m \) silent-bridges to \( j\sigma' \), or
\item there exists states \( j\sigma_{lio} \), \( j\sigma_{lio}' \) and \( j\sigma_m \) such that \( j\sigma \) becomes \( j\sigma_{lio} \) (last input/output) in 0 or more basic-bridges, \( j\sigma_{lio} \) becomes \( j\sigma_{lilo}' \) in exactly 1 basic-bridge whose event is an input or output, \( j\sigma_{lio}' \) becomes \( j\sigma_m \) in 0 or more write-bridges, and \( j\sigma_m \) silent-bridges to \(j\sigma' \) 
\end{itemize}
In other words, in the first case there are no public inputs or output produced at all, and in the second case we separate the last public input or output. This is done because to prove PINI, we must show that any two runs that have produced the same amount of inputs or outputs (even if still mid-execution) have identical public traces, so separating the last public input or output is convenient in the proof. 

%We define the transitive closure for the write-bridge relation. One may notice that it cannot be
%reflexive, as it imposes at least one step emitting a write event.

%The bridge relation (\emph{bridge} in the Coq development) is our main bridging relation. There is
%two different ways for two states to bridge:
%first, the starting state takes any number of basic-bridges, then it takes one basic-bridge that do
%not emit a write-event, then any number of write-bridge, and finally any number of silent-bridges;
%the second way is to take any number of write-bridge, and then any number of silent-bridges.

Figure~\ref{fig:bridge_relation} illustrates the adequacy theorem below, and shows these two different ways to bridge, where $b$ stands for
basic-bridge, $bnw$ for basic-bridge non-write, $wb$ for write-bridge and $sb$ for silent-bridge.
The adequacy theorem below allows one to move from the original language to the auxiliary one, and the other way around.

% Motivation for all the relations
%The different bridge relations allows us to decompose any pack of steps by the kind of events, and
%to know precisely what is synchronized or not each time.
%\instruction{a bit more motivations: the first path actually change the trace, while the 2nd one not
%really, as it only perform public writes}

We can derive two main theorems: an adequacy theorem and an agreement theorem.
\begin{theorem}[Bridge adequacy]\label{thm:adequacy}
  Let $\conf \triangleq (c,~S,~P,~\tau)$ and $\conf' \triangleq (c',~S',~P',~\tau')$ be two
  configurations of \whilelang, $\ctx$ and $\ctx'$ be resp. an initial and final context,
  and $\pc$ be an initial pc.
  Let further suppose that

 \begin{itemize}
   \item $m$ be a well-formed memory wrt. $\ctx$,
   \item \(\typing{\ctx}{\pc}{c}{\ctx'} \)
   \item \( \execn{\conf}{n}{\conf'} \)
 \end{itemize}

  Then, it exists $j'$ and $\tau_{f}$ such that \( \bridge{\compile{c}}{\tau_{f}}{j'} \), $c' = \uncompile{j'}$ and \( \tau_f \) becomes \( \pproj{\tau'} \) when removing the \( \mathrm{Write}~x~v \) and \( \mathrm{BecameSecret}~x \) events.
\end{theorem}
\instruction{overview of the proofs ?}

In other words, the \emph{bridge adequacy} theorem states that if a command is well-typed, and takes
any number of steps, then its auxiliary counterpart bridges to a state that is the compilation of
the final state and that records exactly the public trace.
% \instruction{I'm not happy with this explanation, find a better formulation}

This theorem then allows to reason on the bridge relation rather than on the operational semantic.
The bridge relation decomposes the steps by focusing only on the emitted public events, and gives
some more information about the states that are synchronized (or not).
Figure~\ref{fig:bridge_relation} draws the adequacy.

\begin{figure}[ht]
  \centering

  \begin{tikzcd}
    compile(c) \arrow[r, "b*"]                                          & c0' \arrow[r, "bnw"] & c1' \arrow[r, "wb*"] & c2' \arrow[r, "sb"]          & c3' \arrow[d, "adequacy"]   \\
    c \arrow[rrrr, "exec"] \arrow[u, "adequacy"] \arrow[d, "adequacy"'] &                      &                      &                        & c'                          \\
    compile(c) \arrow[rrr, "wb*"']                                      &                      &                      & c2'' \arrow[r, "sb"'] & c3'' \arrow[u, "adequacy"']
  \end{tikzcd}
  \caption{\label{fig:bridge_relation} Bridge relation }
\end{figure}


\begin{theorem}[Bridge agree]\label{thm:agree}
  If two observationally equivalent states bridge to two final states, where the length of the public projection of trace in the final states are equal, then those public projections are equal
  
  %Assume
%  \begin{itemize}
 %   \item $\agree{\ctx}{m_{1}}{m_{2}}$
  %  \item $\typing{\ctx}{\pc}{c}{\ctx_{f}}$
   % \item $\pproj{\tau_{1}} = \pproj{\tau_{2}}$
%    \item $\bridge{(c,~S_{1},~P,~m_{1},~\tau_{1})}{n_{1}}(c_{1}',~S_{1}',~P_{1}',~m_{1}',~\tau_{1}')$
 %   \item $\bridge{(c,~S_{2},~P,~m_{2},~\tau_{2})}{n_{2}}(c_{2}',~S_{2}',~P_{2}',~m_{2}',~\tau_{2}')$
  %  \item $\mathrm{len}(\pproj{\tau_{1}'}) = \mathrm{len}(\pproj{\tau_{2}'})$
%  \end{itemize}

 % Then,
%  the final commands are the same $c_{1}' = c_{2}'$,
%  the memories agree on the public variables $\agree{\ctx_{f}}{m_{1}}{m_{2}}$, % is this true ???
%  and the public traces are the same $\pproj{\tau_{1}'} = \pproj{\tau_{2}'}$
\end{theorem}

The theorem can be split in 3 lemmas: basic-bridge agree, write-bridge agree, and silent-bridge
agree. These state stronger properties: if two observationnally equivalent states basic-bridge or write-bridge to two final states, those final states are observationally equivalent; and if an initial state silent-bridges to a final state, then that initial state and that final state are observationally equivalent. These lemmas can be derived from the observation we made earlier on single execution steps.

%Silent-bridge agree states that in-between two silent bridge of two executions,
%the memory agree on public, public input are the same and the public
%traces are the same for both execution. It holds because silent-bridge relates states that do not
%emit any public events: the traces don't change, and the public part of the memory don't change
%either.

%Write-bridge agree states that in-between two write-bridges of two executions,
%the public traces are the same for both execution. It holds because write-bridge relates states that
%do not emit only write-events, which means that memories might not agree on the public anymore, but
%they do not emit any visible events in the trace.

%Basic-bridge is the main lemma: it states that in-between two basic-bridges of two executions of
%same length, the final command is the same, the final memory agree on public, and
%public traces are the same for both execution.
%It holds because \instruction{fill in}.

%\instruction{overview of the proofs ?}

\subsection{Main theorem}%
\label{subsec:main_theorem}

The soundness theorem mainly relies on the following theorem. It states that for
any pair of configurations that agree on public of the memory, agree on the public events,
that start with a well-typed command, and both step into a configuration that produces a
public trace of the same length actually produces the same public trace.

\begin{theorem}[Well-typed execution agree]\label{thm:sound}

Suppose that

\begin{enumerate}
  \item $m_{1}$ and $m_{2}$ are well-formed wrt. $\ctx$
  \item $\agree{\ctx}{m_{1}}{m_{2}}$
  \item \( \typing{\ctx}{\mathrm{Public}}{c}{\ctx_{f}}\)
  \item $\pproj{\tau_{1}} = \pproj{\tau_{2}}$
  \item \(
        \exectrans
        {(c,~S_{1},~P,~m_{1},~\tau_{1})}
        {(c,~S_{1}',~P_{1}',~m_{1}',~\tau_{1}')}
        \)
  \item \(
        \exectrans
        {(c,~S_{2},~P,~m_{2},~\tau_{2})}
        {(c,~S_{2}',~P_{2}',~m_{2}',~\tau_{2}')}
        \)
  \item $\mathrm{len}(\pproj{\tau_{1}'}) = \mathrm{len}(\pproj{\tau_{2}'})$
\end{enumerate}

Then $\pproj{\tau_{1}'} = \pproj{\tau_{2}'}$.
\end{theorem}


\begin{proof} (Sketch)

  The proof of this theorem is in essence a composition of the previous theorems. The main idea is
  that, using the Theorem~\ref{thm:adequacy} in the hypotheses 5 and 6, we can reason at the level
  of the bridge relation, rather than at the level of the operational semantic itself.
  Then, using Theorem~\ref{thm:agree}, we know that the emitted public traces are the same in both
  executions.

\end{proof}

Finally, we can define the soundness theorem that state that any well-typed command is PINI.

\begin{theorem}[Soundness type-system]
  \( \forall c, \ctx.\
  \typing{\emptyset}{\mathrm{Public}}{c}{\ctx}
  \Rightarrow \pini{c}
  \)
\end{theorem}

\begin{proof} (Sketch)

  Let $c$ be a command and $\ctx$ be a context. Assume that $c$ is well-typed under the empty
  context, changing it to $\ctx$.

  By unfolding the definition of PINI, we assume a public stream $P$, an already known trace
  $\delta$ and any public event $\alpha$. We also assume that we have a first execution of the
  program with some secret input stream $S_{1}$, which produces a trace $\tau_{1}$ such that
  $\pproj{\tau_{1}} = \alpha::\delta$.
  \instruction{double check that I'm not wrong}

  We need to prove \(K(P,~c,~\alpha::\delta) = K_{\rightarrow}(P,~c,~\delta) \).
  The left-to-right direction is almost trivial, as we already know the next event $\alpha$.
  The right-to-left direction is the hard one. By unfolding the definition, we get that
  the second execution emits a trace $ \pproj{\tau_{2}} = \alpha_{2} :: \delta$.
  The proof thus mainly relies on the fact that $\pproj{\tau_{1}}= \pproj{\tau_{2}}$.

  Using the Theorem \ref{thm:sound}, all the condition are met, and we can conclude the proof.

\end{proof}


\section{Limits and conclusion}%
\label{sec:conclusion}

We have formalized the type-system in Coq, but the formalization does not include the whole report.

\paragraph{Incomplete type-system}
The type-system that we defined actually does not include the typing rule for the While
command.
The main reason is that, to prove the \emph{preservation lemma}~\ref{thm:preservation}, we need an
intermediate lemma\footnotemark[2] stating that, for all well-typed command, (1) the starting typing
context can flows from a more relaxed typing context and (2) the ending typing context can flows to
a stricter typing context. This is required, for instance, to prove the If case, where the exit
context is flows from the two inductive cases (as it is a join).
However, the aforementioned lemma seems to not be true in the case of While. Indeed, the inductive
case gives a typing context that is more relaxed than the typing context used for the body of the
loop, but it has not necessarily any relation with the typing context of the While.
\footnotetext[2]{In the Coq development, the lemma is named typecheck\_flow\_gen}

\paragraph{Incomplete proof}
Even without the While typing rule, we actually didn't manage to terminate the proof.
The admitted lemmas are the following:
\instruction{list the admitted lemmas}
Although we didn't prove them, we have good reason to believe that we could have go through the
whole proof if we had more time to complete them. The Coq implementation explains in the comment why
we think that the missing lemmas are provable.

\paragraph{Conclusion}
In this work, we design a type-system for the \whilelang that captures PINI.
The type-system is mechanized in Coq, as well as its proof of soundness.

\appendix

\section{Examples}%
\label{sec:examples}

In this section, we present some interesting program that are well-typed according to the
type-system presented in Figure~\ref{fig:typing_cmd}.

\paragraph{Motivating While}
The following example motivates the while rule, with final context that flows from the initial
context. In particular, the initial context at the beginning of the loop is
$\ctx = \{ x \mapsto \public,\ y \mapsto \public \}$, whereas the final context, when leaving the
loop, is $\ctx' = \{ x \mapsto \secret,\ y \mapsto \public \}$. In particular, we remark that after
unrolling the loop once, $x$ is secret, which change the pc of the body from public to secret.
\begin{Verbatim}[samepage=true]
input(public, x) ;
input(public, y) ;
while x:
do
   input(secret, x);
done
output(public, y);
\end{Verbatim}

\paragraph{Example 1}
\begin{Verbatim}[samepage=true]
input(secret, x) ;
x := 0 ;
output(public, 0)
\end{Verbatim}

\paragraph{Example 2}
\begin{Verbatim}[samepage=true]
input(secret, x) ;
if x then y else skip ;
output(secret, y);
y := 0 ;
output(public, y)
\end{Verbatim}

\paragraph{Example 3}
\begin{Verbatim}[samepage=true]
input(secret, x) ;
while x:
do
   x := x - 1 ;
   input(y, secret) ;
done
output(public, 0)
\end{Verbatim}

\paragraph{Example 4}
\begin{Verbatim}[samepage=true]
input(secret, x) ;
input(public, x) ;
output(public, x)
\end{Verbatim}


\section{Operational Semantic \whilelang}%
\begin{figure}[ht]

  \centering
 \begin{mathparpagebreakable}
    \inferH{Expr-Lit} { }
    {\langle n, m \rangle \Downarrow n}
    \and

    \inferH{Expr-Var}
    {m(x) = v}
    {\langle x, m \rangle \Downarrow v}
    \and

    \inferH{Expr-BinOp}
    { \langle e_{1}, m \rangle \Downarrow v_{1} \and
      \langle e_{2}, m \rangle \Downarrow v_{2} \and
    }
    {\langle e_{1} \oplus e_{2}, m \rangle \Downarrow v_{1} \oplus v_{2}}
    \and


    \inferH{Skip}
    { }
    { \exec{(\mathrm{Skip},~S,~P,~m,~\tau)}{(\mathrm{stop},~S,~P,~m,~\tau)} }
    \and

    \inferH{Assign}
    { \langle e, m \rangle \Downarrow v }
    { \exec{(\mathrm{Assign}~x~e,~S,~P,~m,~\tau)}{(\mathrm{stop},~S,~P,~m[x \mapsto v],~\tau)} }
    \and

    \inferH{Seq1}
    {
      \exec{(c_{1},~S,~P,~m,~\tau)}{(c_{1}',~S',~P',~m',~\tau')}
      \and
      c_{1} \neq \mathrm{stop}
    }
    {\exec{(c_{1};;c_{2},~S,~P,~m,~\tau)}{(c_{1}';;c_{2},~S',~P',~m',~\tau')}}
    \and

    \inferH{Seq2}
    {
      \exec{(c_{1},~S,~P,~m,~\tau)}{(\mathrm{stop},~S',~P',~m',~\tau')}
    }
    {\exec{(c_{1};;c_{2},~S,~P,~m,~\tau)}{(c_{2},~S',~P',~m',~\tau')}}
    \and

    \inferH{If}
    { \langle e, m \rangle \Downarrow v
      \and
      i = (\mathrm{if\ v \neq 0\ then\ 1\ else\ 2)}
    }
    {\exec{(\mathrm{If}~e~c_{1}~c_{2},~S,~P,~m,~\tau)}{(c_{i},~S,~P,~m,~\tau)}}
    \and

    \inferH{While}
    { }
    {\exec{(\mathrm{While}~e~c,~S,~P,~m,~\tau)}
      {(\mathrm{If}~e~(c;;{While}~e~c)~\mathrm{Skip},~S,~P,~m,~\tau)}}
    \and

    \inferH{Input-Public}
    { P = v :: P' }
    {\exec{(\mathrm{Input}~\mathrm{Public}~x,~S,~P,~m,~\tau)}
      {(\mathrm{stop},~S,~P',~m[ x \mapsto v ],~(\mathrm{Input}(v, \mathrm{public}))::\tau)}}
    \and

    \inferH{Input-Secret}
    { S = v :: S' }
    {\exec{(\mathrm{Input}~\mathrm{Secret}~x,~S,~P,~m,~\tau)}
      {(\mathrm{stop},~S',~P,~m[ x \mapsto v ],~(\mathrm{Input}(v, \mathrm{secret}))::\tau)}}
    \and


    \inferH{Output}
    { \langle e, m \rangle \Downarrow v }
    {\exec{(\mathrm{Output}~ch~x,~S,~P,~m,~\tau)}
      {(\mathrm{stop},~S,~P,~m,~(\mathrm{Output}(v, ch))::\tau)}}
 \end{mathparpagebreakable}

\caption{\label{fig:lang_opsem} Operational semantic of \whilelang}
\end{figure}




\section{Auxilliary language formalization}%
\begin{figure}[ht]
  \centering
  \begin{mathparpagebreakable}
    \inferH{Skip}
    { }
    {
      \execaux
      {(\mathrm{Skip},~S,~P,~m,~\tau)}{\ctx}{\lpc}
      {\varepsilon}
      {(\mathrm{stop},~S,~P,~m,~\tau)}{\ctx}{\lpc}
    }
    \and

    \inferH{Assign}
    {
      \langle e, m \rangle \Downarrow v \and
      \etyping{\ctx}{e}{\ell} \and
      \alpha = \mathrm{(if\ \ell = LPublic\ then\ Write(x,v)\ else\ \varepsilon)}
    }
    {
      \execaux
      {(\mathrm{Assign}~x~e,~S,~P,~m,~\tau)}{\ctx}{\lpc}
      { \alpha }
      {(\mathrm{stop},~S,~P,~m,~\tau)}{\ctx[x \mapsto (\join{\lpc}{\ell})]}{\lpc}
    }
    \and

    \inferH{Seq-1}
    {
      \execaux
      {(c_{1},~S,~P,~m,~\tau)}{\ctx}{\lpc}
      {\alpha}
      {(c_{1}',~S',~P',~m',~\tau')}{\ctx'}{\lpc'}
      \and
      c_{1} \neq \mathrm{stop}
    }
    {
      \execaux
      {(c_{1};;c_{2},~S,~P,~m,~\tau)}{\ctx}{\lpc}
      {\alpha}
      {(c_{1}';;c_{2},~S',~P',~m',~\tau')}{\ctx'}{\lpc'}
    }
    \and

    \inferH{Seq-2}
    {
      \execaux
      {(c_{1},~S,~P,~m,~\tau)}{\ctx}{\lpc}
      {\alpha}
      {(\mathrm{stop},~S',~P',~m',~\tau')}{\ctx'}{\lpc'}
    }
    {
      \execaux
      {(c_{1};;c_{2},~S,~P,~m,~\tau)}{\ctx}{\lpc}
      {\alpha}
      {c_{2},~S',~P',~m',~\tau')}{\ctx'}{\lpc'}
    }
    \and

    \inferH{If}
    {
      \langle e, m \rangle \Downarrow v \and
      \etyping{\ctx}{e}{\ell} \and
      i = (\mathrm{if\ v \neq 0\ then\ 1\ else\ 2)}
    }
    {
      \execaux{(\mathrm{If}~e~c_{1}~c_{2},~S,~P,~m,~\tau)}{\ctx}{\lpc}
      {\varepsilon}
      {(c_{i},~S,~P,~m,~\tau)}{\ctx'}{\lpc'}
    }
    \and

    \inferH{Join-1}
    {
      \execaux{(c,~S,~P,~m,~\tau)}{\ctx}{\lpc}
      {\alpha}
      {(c',~S',~P',~m',~\tau')}{\ctx'}{\lpc'}
      \and
      c' \neq \mathrm{stop}
    }
    {
      \execaux{(\mathrm{ThenJoin}~c,~S,~P,~m,~\tau)}{\ctx}{\lpc}
      {\alpha}
      {(\mathrm{ThenJoin}~c',~S',~P',~m',~\tau')}{\ctx'}{\lpc'}
    }
    \and

    \inferH{Join-2}
    {
      \execaux{(c,~S,~P,~m,~\tau)}{\ctx}{\lpc}
      {\alpha}
      {(\mathrm{stop},~S',~P',~m',~\tau')}{\ctx'}{\lpc'}
    }
    {
      \execaux{(\mathrm{ThenJoin}~c,~S,~P,~m,~\tau)}{\ctx}{\lpc}
      {\alpha}
      {(\mathrm{stop},~S',~P',~m',~\tau')}{\ctx'}{\lpc'}
    }
    \and


    \inferH{While}
    { }
    {
      \execaux{(\mathrm{While}~e~c,~S,~P,~m,~\tau)}{\ctx}{\lpc}
      {\varepsilon}
      {(\mathrm{If}~e~(c;;{While}~e~c)~\mathrm{Skip},~S,~P,~m,~\tau)}{\ctx}{\lpc}
    }
    \and

    \inferH{Input-Public}
    {
      \ctx' = \ctx[ x \mapsto \lpc ] \and
      \tau' = (\mathrm{Input}(v, \mathrm{public}))::\tau
    }
    {
      \execaux{(\mathrm{Input}~\mathrm{Public}~x,~S,~(v :: P),~m,~\tau)}{\ctx}{\lpc}
      {\mathrm{Input v}}
      {(\mathrm{stop},~S,~P,~m[ x \mapsto v ],~\tau')}{\ctx'}{\lpc}
    }
    \and

    \inferH{Input-Secret}
    {
      \ctx' = \ctx[ x \mapsto \secret ] \and
      \tau' = (\mathrm{Input}(v, \mathrm{secret}))::\tau
    }
    {
      \execaux{(\mathrm{Input}~\mathrm{Secret}~x,~(v :: S),~P,~m,~\tau)}{\ctx}{\lpc}
      {\varepsilon}
      {(\mathrm{stop},~S,~P,~m[ x \mapsto v ],~\tau')}{\ctx'}{\lpc}
    }
    \and

    \inferH{Output}
    {
      \langle e, m \rangle \Downarrow v \and
      \etyping{\ctx}{e}{\ell} \and
      \flowsto{\join{\lpc}{\ell}}{ch} \and
      \alpha = \mathrm{Output}(v, ch) \and
      ev = \mathrm{(if\ ch = Public\ then\ \alpha\ else\ \varepsilon)}
    }
    {
      \execaux{(\mathrm{Output}~ch~x,~S,~P,~m,~\tau)}{\ctx}{\lpc}
      {ev}
      {(\mathrm{stop},~S,~P,~m,~\alpha::\tau)}{\ctx}{\lpc}
    }

  \end{mathparpagebreakable}

  \caption{\label{fig:lang_aux_opsem} Operational semantic of the auxiliary language}
\end{figure}

\begin{figure}[ht]
  \centering
 \begin{mathparpagebreakable}
   \inferH{Skip}
   { \flowsto{\ctx}{\ctx_{f}} \and \flowsto{\lpc}{\lpc_{f}} }
   { \typingaux{\ctx}{\lpc}{\mathrm{Skip}}{\ctx_{f}}{\lpc_{f}} }
   \and

    \inferH{Assign}
    { \etyping{\ctx}{e}{ \ell}
      \and
      \flowsto{\ctx[x \mapsto \join{\lpc}{\ell}]}{\ctx_{f}}
      \and \flowsto{\lpc}{\lpc_{f}}
    }
    { \typingaux{\ctx}{\lpc}{\mathrm{Assign}~x~e}{\ctx_{f}}{\lpc_{f}} }
    \and

    \inferH{Seq}
    {
      \typingaux{\ctx_{1}}{\pc}{c_{1}}{\ctx_{2}} \and
      \typingaux{\ctx_{2}}{\pc}{c_{2}}{\ctx_{3}}
    }
    { \typingaux{\ctx_{1}}{\lpc}{c_{1};;c_{2}}{\ctx_{3}}{\lpc_{f}} }
    \and

    \inferH{If}
    { \etyping{\ctx}{e}{ \ell } \and
      \typingaux{\ctx}{\ell :: \lpc }{c_{1}}{\ctx_{1}}{\lpc_{1}} \and
      \typingaux{\ctx}{\ell :: \lpc }{c_{2}}{\ctx_{2}}{\lpc_{2}} \and
    }
    { \typingaux{\ctx}{\lpc}{\mathrm{If}~e~c_{1}~c_{2}}
      {\join{\ctx_{1}}{\ctx_{2}}}
      {\join{\lpc_{1}}{\lpc_{2}}} }
    \and

    \inferH{While}
    { \flowsto{\ctx}{\ctx_{f}} \and
      \flowsto{\lpc}{\lpc_{f}} \and
      \etyping{\ctx_{f}}{e}{ \ell } \and
      \typingaux{\ctx_{f}}{\ell :: \pc}{c}{\ctx_{f}}{\lpc_{f}}
    }
    { \typingaux{\ctx}{\lpc}{\mathrm{While}~e~c}{\ctx_{f}}{\lpc_{f}} }
    \and

    \inferH{Input}
    { \flowsto{\lpc}{ch} \and
      \flowsto{\lpc}{\lpc_{f}} \and
      \flowsto{\ctx[ x \mapsto \join{\lpc}{ch}]}{\ctx_{f}}
    }
    { \typingaux{\ctx}{\lpc}{\mathrm{Input}~ch~x}{\ctx_{f}}{\lpc_{f}} }
    \and

    \inferH{Output}
    { \etyping{\ctx}{e}{\ell} \and
      \flowsto{\ell::\pc}{ch} \and
      \flowsto{\ctx}{\ctx_{f}} \and
      \flowsto{\lpc}{\lpc_{f}}
    }
    { \typingaux{\ctx}{\lpc}{\mathrm{Output}~ch~e}{\ctx_{f}}{\lpc_{f}} }
    \and

    \inferH{Join}
    { \typingaux{\ctx}{\lpc}{c}{\ctx_{f}}{\ell::\lpc_{f}} }
    { \typingaux{\ctx}{\lpc}{\mathrm{ThenJoin}~c}{\ctx_{f}}{\lpc_{f}} }
 \end{mathparpagebreakable}
  \caption{\label{fig:typing_aux} Typing rules for the auxiliary language }
\end{figure}


%% References
%% - Edit literature.bib to add references
%%
% \bibliographystyle{abbrvnat}
% \bibliography{literature}

\end{document}
